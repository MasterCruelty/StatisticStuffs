\documentclass{article}
%-----------------------
% Document information
%-----------------------


\title{A few exercises}

\author{Roberto \textsc{Antoniello}}

\begin{document}

\maketitle

\begin{center} In this file I put a couple of exercises around the Probability calculus.\end{center}

\section{Delicate question}

There are 100 students and 100 sphere with a number. Every student draws a sphere and must answer with yes or no.

\bigskip

Sphere with number $\leq 70 \rightarrow$ delicate question.

Sphere with number $> 70 \rightarrow$ check question.

\bigskip

We would like to know the probability that an answer is "yes" considering that it's a delicate question. Let's assume we have 25 yes.

$$P(yes) = P(yes|delicate)\cdot P(delicate) + P(yes| check) \cdot P(check)$$

$$P(yes) \approx 0.25 \ \ \ P(yes|check)  = 0.5 \ \ \ P(check) = 0.3 \ \ \ P(delicate) = 0.7$$

$$P(yes|delicate) = \frac{0.25 - (0.5\cdot 0.3)}{0.7} \approx 0.14$$

\section{Books on a rack}

We have 10 books and we would like to know every possible ways to put them on a rack in order by subject.
The books are: $4M + 3C + 2S + 1L$

\bigskip

So we are trying to do permutation over these books and then here the answer:

$$(4!\cdot3!\cdot2!\cdot1!)\cdot 4!$$

I supposed the subject order and I multiplied for the number of possible orders(First M, First C, etc.etc.).

\section{Extraction of balls}

We have 11 balls, 5 of them are white and 6 of them are black. We would like to know the probability of extracting two balls with different colour eachother with two consecutive extractions.

\bigskip

Possible cases $\rightarrow d_{11,2} = 11 \cdot 10 = 110$

Favourable cases $\rightarrow 6 \ \ white \cdot 5 \ \ black = 30$ and it's the same if I extract black before the white.

\bigskip

$$P(two \ \ \ different \ \ \ colours) = \frac{60}{110} = \frac{6}{11}$$

\section{Disease test}

We have a test that tells us if a person has a particolar disease or not. This test get the right answer in 99\% of the cases, it gives a fake positive in 1\% of the cases.

We would like to know the probability of being sick considering I had a positive response of the test.

\bigskip

Let's define two events: 

E = positive

S = sick

\bigskip

99\% is ok

1\% is fake positive

\bigskip

$P(E|M) = 0.99$

$P(E|\bar M) = 0.01$

$P(\bar E| \bar M) = 1 - P(E | \bar M) = 0.99$

$P(M) = 0.005$ (We suppose this as sickness rate)


$$P(M|E) = \frac{P(M \cap E)}{P(E)} = \frac{P(E\cap M)}{P(E)} = \frac{P(E|M)\cdot P(M)}{P(E)} = $$

$$\frac{P(E|M)\cdot P(M)}{P(E|M) \cdot P(M) + P(E|\bar M) \cdot P(\bar M)} = \frac{0.99 \cdot 0.005}{0.99 \cdot 0.005 + 0.01 \cdot 0.995} \approx 0.3322$$

\section{Lost plane}

There's a lost plane and we have 3 zones to search.
We would like to know the probability that the plane has fallen in each zones considering that a research in zone 1 had a negative outcome.

\bigskip

So: $$P(Z_1) = \frac{1}{3} \ \ P(Z_2) = \frac{1}{3} \ \ P(Z_3) = \frac{1}{3}$$

\bigskip

I define $\alpha_i = P(Not \ \ found| I \ \ search \ \ in \ \ Z_i)$

\bigskip

Let's consider the E event that happens when the plane is not found in $Z_1$.

\bigskip

So: $$ P(E|Z_1) = \alpha_1 \ \ P(E|Z_2) = 1 \ \ P(E|Z_3) = 1$$ 


$$P(E) = \sum_{i=1}^3 P(E|R_i)P(R_i) = \frac{\alpha_i}{3} + \frac{2}{3} \ \ Total \ \ probability \ \ theorem$$

$$P(R_1|E) = \frac{P(E| R_1) \cdot P(R_1)}{P(E)} =  \frac{\frac{\alpha_i}{3}}{\frac{(\alpha_1 + 2)}{3}} = \frac{\alpha_1}{\alpha_1 + 2} \ \ Bayes \ \ Theorem$$

For i = 2,3 we obtain: $$\frac{1}{\alpha + 2}$$ 

\section{Collection cards}

There are n collection cards. 

$P_i = P(I \ \ buy \ \ the \ \ i-card)$

\bigskip

$0 \leq P_i \leq 1$

\bigskip

$\sum_{i=1}^k P_i = 1$

\bigskip

I have k cards.

We would like to calculate this probability:

$$P(At \ \ least \ \ a \ \ j \ \ card | I \ \ have \ \ a \ \ i \ \ card)$$

$$P(A_j | A_i ) = \frac{P(A_i \cap A_j)}{P(A_i)} = \frac{1 - (1 - P_i)^k - (1 - P_j)^k + (1 - P_i - P_j)^k}{1 - (1 - P_i)^k}$$

Demonstration: 

\bigskip

First we determine $P(A_i)$

$$P(A_i) = 1 - P(\bar A_i) = 1 - P(I \ \ never \ \ get \ \ the \ \ i-card \ \ on \ \ k \ \ purchases)$$

$$= 1 - P(\bigcap_{i=1}^k \ \ I \ \ don't \ \ get \ \ the \ \ i-card \ \ on \ \ the \ \ k \ \ purchase)$$

$$= 1 - \prod_{i=1}^k P(I \ \ don't \ \ get \ \ the \ \ i-card \ \ on \ \ the \ \ k \ \ purchase)$$

$$= 1 - \prod_{i=1}^k (1 - P_i) = 1 - (1 - P_i)^k$$

\bigskip

Second we determine $P(A_i \cap A_j)$

$$P(A_i \cap A_j) = 1 - P(\overline{A_i \cap A_j}) = 1 - P(\bar A_i \cup \bar A_j)$$

$$= 1 - (P(\bar A_i) + P(\bar A_j) - P(\bar A_i \cap \bar A_j))$$

$$= 1 - ((1 - P_i)^k + (1 - P_j)^k - P(\bar A_i \cap \bar A_j))$$

\bigskip

To finish we have to determine $P(\bar A_i \cap \bar A_j)$

$$P(\bar A_i \cap \bar A_j) = P(I \ \ never \ \ get \ \ the \ \ i-card \ \ and \ \ j-card \ \ on \ \ k \ \ purchases)$$

$$= P(\bigcap_{i=1}^k \ \ I \ \ don't \ \ get \ \ the \ \ i-card \ \ and \ \ j-card \ \ on \ \ the \ \ k \ \ purchase)$$

$$= \prod_{i=1}^k (1- P_i - P_j) = (1 - P_i - P_j)^k$$

\bigskip

So we have determinated that:

$$P(A_i \cap A_j) = 1 - (1 - P_i)^k + (1 - P_j)^k - (1 - P_i - P_j)^k$$

\bigskip

Demonstrated.

\end{document}
